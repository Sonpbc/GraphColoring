%\documentclass{beamer}
\documentclass[mathserif,10pt]{beamer}

\usepackage{beamerthemesplit}
\usepackage{graphics}
\usepackage{epsfig}
\usepackage{algorithm}
\usepackage{verbatim}
\usepackage{listings}
\usepackage{framed}
\usepackage{pstricks}
\usepackage{pst-node,pst-tree}
\usepackage{pst-rel-points}
\usepackage{flexiprogram}
\usepackage[UKenglish]{babel}
\usepackage{hyperref}
\usepackage{pst-coil}
\usepackage{color}
\usepackage{epsfig}
\usepackage{tikz}
%\usepackage{multirow}

\usefonttheme{serif}

\newcommand{\cmt}[1]{}
\newcommand{\epsilonset}{\ensuremath{\{\epsilon\}}}
\newcommand{\epsilonpairset}{\ensuremath{\{\epsilon,\epsilon\}}}
\newcommand{\num}[1]{\ensuremath{|#1|}}
\newcommand{\upath}{\ensuremath{\mathcal{U}}}
\newcommand{\mb}[1]{\mbox{{\tt #1}}}
\newcommand{\ttf}[1]{{\tt #1}}
\newcommand{\rtarrow}{$\rightarrow$}
\newcommand{\Tree}{{\tt Tree}}
\newcommand{\Dag}{{\tt Dag}}
\newcommand{\Cycle}{{\tt Cycle}}
\newcommand{\p}{\ensuremath{p}}
\newcommand{\q}{\ensuremath{q}}
\newcommand{\s}{\ensuremath{s}}
\newcommand{\myr}{\ensuremath{r}}
\newcommand{\shape}{\mbox{shape}}
\newcommand{\drct}{\ensuremath{D}}
\newcommand{\indrct}{\ensuremath{I}}
\newcommand{\heap}{\ensuremath{\mathcal{H}}}
\newcommand{\fields}{\ensuremath{\mathcal{F}}}
\newcommand{\DFM}[2]{\ensuremath{D_F[#1,#2]}}
\newcommand{\IFM}[2]{\ensuremath{I_F[#1,#2]}}
\newcommand{\nat}{\ensuremath{\mathcal{N}}}
\newcommand{\fieldD}[2]{\ensuremath{{#1}_{#2}^\drct}}
\newcommand{\fieldI}[3]{\ensuremath{{#1}_{#2}^{\indrct#3}}}
\newcommand{\subC}{\mbox{\scalebox{0.6}{\Cycle}}}
\newcommand{\subD}{\mbox{\scalebox{0.6}{\Dag}}}
\newcommand{\false}{\textbf{False}}
\newcommand{\true}{\textbf{True}}
%\noindent

\setcounter{tocdepth}{1}
\lstset{language=[ANSI]C}
\lstset{% general command to set parameter(s)
basicstyle=\footnotesize\tt, % print whole listing small
identifierstyle=, % nothing happens
commentstyle=\color{red}, % white comments
showstringspaces=false, % no special string spaces
lineskip=1pt,
captionpos=b,
frame=single,
breaklines=true
%\insertauthor[width={3cm},center,respectlinebreaks]
}
\lstset{classoffset=0,
morekeywords={},keywordstyle=\color{black},
classoffset=1,
classoffset=0}% restore default

%\usetheme{Warsaw}
\usetheme{CambridgeUS}
%\usetheme{Antibes}
%\usecolortheme{lily}
%\useinnertheme{rectangles} 
%\useoutertheme{infolines} 
%\setbeamercolor{alerted text}{fg=cyan}
%\beamertemplatetransparentcovereddynamicmedium
%\definecolor{bbrown}{rgb}{.6588,.4,.1647}
%\definecolor{blueviolet}{rgb}{.098039216,.050980392,.929411765}
%\definecolor{periwinkle}{rgb}{.423529412,.458823529,.988235294}
%\mode<presentation>
%{ \usetheme{boxes} }
\usecolortheme{dolphin}

\title{Graph Coloring Using State Space Search}
\author[Presented by: Group 1]{{\textbf{Sandeep, Tanmay, Mengjia, Novi}} }
\date{December 09, 2014}
\begin{document}

\begin{frame}
\titlepage
\end{frame}
\usebeamertemplate{mytheme}

\AtBeginSection[]
{
\begin{frame}<beamer>
\frametitle{Outline}
\tableofcontents[currentsection]
\end{frame}
}

\section{Introduction}
\subsection{Introduction}
\frame
{
	\frametitle{\subsecname}
	\uncover<1>{\textbf{Shape Analysis : } Class of techniques that statically approximate the
      run-time structures created on the heap.}
}

\cmt{ 
\subsection{Application}
\frame
{
	\frametitle{\subsecname}
		
	\begin{figure}
  	\begin{center}
  
    \scalebox{.80}{
		\begin{tabular}[htbp]{ | c | c  |}
    	\hline
    	& \multirow{3}{*}{ \input{code_treeadd_colored_rw}} \\
      \includegraphics[scale=0.6]{tree_grph}
      & \\
     (a) Heap structure & 
     (b Function traversing the data structure. \\
     \hline
    \end{tabular}
	}
  	\end{center}
	\end{figure}

	\begin{center}
	(p.shape $==$ Tree) $\Rightarrow$ {\tt S2} and {\tt S4} can be executed in parallel.
  	\end{center}
}

\subsection{Programming Model}
\frame
{
	\frametitle{\subsecname}
	
\begin{figure}
  \begin{center}
  
    \scalebox{.8}{\begin{tabular}{ | c | c | c | c |}
   \hline
   & Before execution & Statement & After execution \\
   \hline
    Allocations & \input{allocation_before} & \ttf{p = malloc()} & \input{allocation_after}\\
    \hline
    \multirow{3}{*}{Pointer Assignments} & \input{fig_ptrassgn_1} & \ttf{p = q} & \input{fig_ptrassgn_1_after} \\
    \cline{2-4}
    & \input{fig_ptrassgn_2} & \ttf{p = q\rtarrow{f}} & \input{fig_ptrassgn_2_after} \\
    \cline{2-4}
    & \input{allocation_after} & \ttf{p = NULL} & \input{allocation_before} \\
     \hline
     \multirow{2}{*}{Structure updates} & \input{ptr_strup_1} & \ttf{p\rtarrow{f} = NULL} & \input{ptr_strup_1_after} \\
     \cline{2-4}
     & \input{ptr_strup_2} & \ttf{p\rtarrow{f} = q} & \input{ptr_strup_2_after} \\
     \hline
    \end{tabular}}
  \end{center}
\end{figure}
}

\section{Related Work}
\subsection{Related Work}
\frame
{
	\frametitle{\subsecname}
	\begin{itemize}
	\item Work done by Ghiya et. al\footnote{Rakesh Ghiya and Laurie J. Hendren. Is it a Tree, a DAG, or a Cyclic graph? 
	a shape analysis for heap-directed pointers in C. In Proceedings of the 23rd ACM
	SIGPLAN-SIGACT symposium on Principles of programming languages, pages 1-15, January 1996.}
	
	\begin{itemize}
	\item Keeps interference and direction matrices between any two 
			pointer variables.
	\item Infer the shape of the structures as Tree, DAG or Cycle
	\item Conservatively identify the shapes.
	\end{itemize}
	\end{itemize}
}
\frame
{
	\frametitle{\subsecname}
	\begin{itemize}
	\item Work done by Sagiv et. al \footnote{Mooly Sagiv, Thomas Reps, and ReinhardWilhelm. Parametric shape analysis via
	3-valued logic. POPL 1999, pages 105-118, 1999.}
	
	\begin{itemize}
	\item Introduce the concepts of abstraction and re-materialization.
	\item Potentially exponential run-time in the number of predicates.
	\end{itemize}
	\end{itemize}
		
}
\cmt{
\frame
{
	\frametitle{\subsecname}
	\begin{itemize}
	\item Work done by Marron et. al \footnote{Mark Marron, Deepak Kapur, Darko Stefanovic, and Manuel Hermenegildo. A
static heap analysis for shape and connectivity: unified memory analysis: the base
framework. LCPC'06, pages 345-363, 2006.}
	
	\begin{itemize}
	\item Presents a data flow framework that uses heap graphs to model data flow values.
	\item The analysis uses technique similar to re-materialization, but the re-materialization is approximate and 
	may result in loss of precision.
	\end{itemize}
	\end{itemize}
		
}
}

\section{Motivation}
\subsection{Motivation}
\frame
{
	\frametitle{\subsecname}

	\cmt{For each pointer variable, our analysis computes the shape
attribute of the data structure pointed to by the variable.} 
We define the shape attribute $\p.\shape$
for a pointer $\p$ as follows:
%%
\begin{eqnarray*}
  \p.\shape = \left\{ \begin{array}{@{}rl@{}}
    \Cycle & \mbox{If a cycle can be reached from $\p$} \\ 
    \Dag & \mbox{Else if a DAG can be reached from $\p$} \\
    \Tree & \mbox{Otherwise} \\
  \end{array} \right.
\end{eqnarray*}
	\begin{center}
      \includegraphics[scale=0.6]{Figure/figure_1}
	\end{center}
%%
\cmt{
where the heap is visualized as a directed graph, and cycle
and DAG have there natural graph-theoretic meanings.	}
}

\frame
{
	\frametitle{\subsecname}
	
\begin{figure}
  \begin{center}
  
    \scalebox{.8}{\begin{tabular}{ | l | c | c | c |c | c | }
   \hline
   \multirow{2}{*}{Statements} & \multirow{2}{*}{Heap Structure} & \multicolumn{2}{c|}{Actual Shape} &  \multicolumn{2}{c|}{Field Insensitive} \\
   \cline{3-6} 
   &  & p.shape & q.shape & p.shape & q.shape  \\
   \hline
	 &&&&& \\
    {\tt S1.} \ttf{p = malloc()} & \multirow{2}{*}{\input{motiv_1_fig_S1.tex}} &  \multirow{2}{*}{Tree} & \multirow{2}{*}{Tree} & \multirow{2}{*}{Tree} & \multirow{2}{*}{Tree} \\
    {\tt S2.} \ttf{q = malloc()} &  &  &&&\\
	 &&&&& \\
    \hline
	\pause
	 &&&&& \\
	{\tt S3.} \ttf{p$\rightarrow$f = q} &  \input{motiv_1_fig_S3.tex} & Tree & Tree & Tree & Tree\\
	 &&&&& \\
     \hline
	\pause
	 &&&&& \\
	{\tt S4.} \ttf{p$\rightarrow$h = q} &  \input{motiv_1_fig_S4.tex} & Dag & Tree & Dag & Tree \\
	 &&&&& \\
     \hline
    \end{tabular}}
  \end{center}
\end{figure}
}

\frame
{
	\frametitle{\subsecname}
	
\begin{figure}
  \begin{center}
  
    \scalebox{.8}{\begin{tabular}{ | l | c | c | c |c | c | }
   \hline
   \multirow{2}{*}{Statements} & \multirow{2}{*}{Heap Structure} & \multicolumn{2}{c|}{Actual Shape} &  \multicolumn{2}{c|}{Field Insensitive} \\
   \cline{3-6} 
   &  & p.shape & q.shape & p.shape & q.shape  \\
   \hline
	 &&&&& \\
	{\tt S5.} \ttf{q$\rightarrow$g = p} &  \input{motiv_1_fig_S5.tex} & Cycle & Cycle & Cycle & Cycle\\
	 &&&&& \\
     \hline
	\pause
	 &&&&& \\
	{\tt S6.} \ttf{q$\rightarrow$g = null} &  \input{motiv_1_fig_S6.tex} & Dag & Tree & Cycle & Cycle \\
	 &&&&& \\
     \hline
	\pause
	 &&&&& \\
	{\tt S7.} \ttf{p$\rightarrow$h = null} &  \input{motiv_1_fig_S7.tex} & Tree & Tree & Cycle & Cycle \\
	 &&&&& \\
     \hline
    \end{tabular}}
  \end{center}
\end{figure}
}

\frame
{
	\frametitle{\subsecname}
	
	Field insensitive shape analysis algorithms use conservative
	kill information. 
	\cmt{and hence they are, in general, unable to
	infer the shape transition from cycle to DAG or from DAG to
	Tree.}
	\input{motiv_fig_1.tex}
}

\section{Our Analysis}
\subsection{Our Analysis}
\frame
{
	\frametitle{\subsecname}
	\begin{itemize}
	\item {\blue \textbf{Path Length Unbounded}}: \pause Approximate any path between two variables by the first field that is dereferenced on the path. 
	\pause
	\item  {\blue \textbf{Path Count Unbounded}}: \pause Use k-limiting to approximate the number of paths starting at a given field.
	\end{itemize}
}
\frame
{
	\frametitle{\subsecname}
	
	Our analysis remembers the path information using the following:
	\begin{itemize}
	\item  {\blue $D_F$}: Modified direction matrix. \cmt{ that stores the first fields of the paths between two pointers.}
	\pause
	\item {\blue $I_F$}: Modified interference matrix. \cmt{ that stores the pairs of first fields corresponding to the pairs of interfering paths.}
	\pause
	\item {\blue Boolean Variables}: For $f \in \fields, \p, \q \in \heap$, \\ 
			\begin{center}
			$f_{pq}$ == \true\ $ \Rightarrow$  $f$ field of $\p$ points directly to $\q$. 
			\end{center}
	
	\cmt{Field connectivity information: Remember the fields directly connecting two pointer variables.}
	\end{itemize}
}

\cmt{ NOT INCLUDED
\subsection{Notion of Path}
\frame
{
	\frametitle{\subsecname}
	\begin{itemize}
	\item A path from $\p$ to $\q$ is the sequence of pointer fields that need to be traversed in 
			the heap to reach from $\p$ to $\q$.
	\item Path length may be unbounded, so we consider only the first field of a path.
	\item For a path of length one (direct path) ($f_\drct$); For a path of length greater than one
			(indirect path) ($f_\indrct$).
	\item Possible to have multiple indirect paths starting with the same field : k-limiting, beyond k treat the number 
			of paths to be $\infty$.
	\end{itemize}
}
}

\frame
{
	\frametitle{\subsecname}
	\input{defn_fig_0.tex}
	Assuming the limit $k \geq 3$, the path information between $\p$ and $\s$ can be represented by the set:
	$\{g^\drct, f^{\indrct\infty}, h^{\indrct 3} \}$. 
	If $k < 3$, then the set becomes: 
	$\{ g^\drct, f^{\indrct\infty}, h^{\indrct\infty} \}$
}

\cmt{ NOT INCLUDED
\subsection{Direction Matrix : $D_F$}
\frame
{
	\frametitle{\subsecname}
	\begin{definition}
\label{DFM_matrix}
Field sensitive Direction matrix
$D_F$ is a matrix that stores information 
about paths between two pointer variables.
Given $\p, \q \in
\heap, f \in \fields$:
\begin{eqnarray*}
  \epsilon & \in \DFM{p}{p}& \mbox{ where $\epsilon$
    denotes the empty path.} \\
  f^\drct  &\in  \DFM{p}{q} & \mbox{ if there is a direct
    path $f$ from $\p$ to $\q$.}\\
  f^{\indrct m} & \in  \DFM{p}{q} & 
  \mbox{\begin{tabular}[t]{p{70mm}}if there are $m$ indirect
      paths starting with field $f$ from $\p$ to $\q$ and $m
      \leq k.$
    \end{tabular}
  } \\
  f^{\indrct\infty} & \in  \DFM{p}{q} &
  \mbox{\begin{tabular}[t]{p{70mm}}if there are $m$ indirect
      paths starting with field $f$ from $\p$ to $\q$ and $m >
      k.$
  \end{tabular}}  
\end{eqnarray*}
\end{definition}
}

\subsection{Abstraction : $D_F$}
\frame
{
	\frametitle{\subsecname}
	Let \nat\ denote the set of natural numbers. We define the
following partial order for approximate paths used by our
analysis. For $ f \in \fields,\ m,n \in \nat,\ n \leq m$:
$$
\epsilon \sqsubseteq \epsilon, \quad 
f^\drct \sqsubseteq  f^\drct,  \quad
f^{\indrct\infty}  \sqsubseteq  f^{\indrct\infty}, \quad
f^{\indrct m} \sqsubseteq f^{\indrct\infty}, \quad
f^{\indrct n} \sqsubseteq f^{\indrct m} \enspace .
$$
The partial order is extended to set of paths $S_{P_1},
S_{P_2}$ as:
\begin{eqnarray*}
  S_{P_1} \sqsubseteq S_{P_2} &\Leftrightarrow& \forall \alpha \in
  S_{P_1}, \exists \beta \in S_{P_2}\ s.t. \alpha \sqsubseteq \beta \enspace .
\end{eqnarray*}
}

\subsection{Interference Matrix : $I_F$}
\frame
{
	\frametitle{\subsecname}
	\cmt{Two pointers $\p,\q \in \heap$ are said to
interfere if there exists $\s \in \heap$ such that both
$\p$ and $\q$ have paths reaching $\s$. Note that $\s$ could
be $\p$ (or $\q$) itself, in which case the path from $\p$
(from $\q$) is $\epsilon$.}

%FIELD SENSITIVE DIRECTION MATRIX 
\begin{definition}\label{IFM_matrix}
Field sensitive Interference matrix $I_F$ between
two pointers captures the ways in which these pointers are
interfering.  For $\p, \q, \s \in \heap, \p \not= \q$,
the following relation holds for $D_F$ and $I_F$: 
\begin{eqnarray*}
  \DFM{p}{s} \times \DFM{q}{s} &\sqsubseteq&
  \IFM{p}{q} \enspace . \label {eq:rel-df-if}
\end{eqnarray*}
\end{definition}
}

\subsection{Abstraction : $I_F$}
\frame
{
	\frametitle{\subsecname}
	For pair of paths:
\begin{eqnarray*}
  (\alpha, \beta) \sqsubseteq (\alpha', \beta') 
  \Leftrightarrow 
   (\alpha \sqsubseteq \alpha')  \wedge
  (\beta \sqsubseteq  \beta')
\end{eqnarray*}
For set of pairs of paths $R_{P_1}, R_{P_2}$:
\begin{eqnarray*}
  R_{P_1} \sqsubseteq R_{P_2} \Leftrightarrow \forall
  (\alpha, \beta) \in
  R_{P_1}, \exists (\alpha', \beta') \in
  R_{P_2}\ s.t. (\alpha, \beta) \sqsubseteq (\alpha', \beta')
\end{eqnarray*}
}
}

\subsection{Example : $D_F \mbox{ and } I_F$}
\frame
{
	\frametitle{\subsecname}
	\input{defn_fig_1.tex}
}	

\subsection{Boolean Functions}
\frame
{
	\frametitle{\subsecname}
	\begin{itemize}
	\item For each variable $\p \in \heap$, our analysis use attributes $\p_{\subD}$ and $\p_{\subC}$ to store boolean functions. \\
	\begin{center}
			$\p_{\subD} == \true \Rightarrow \p$ can reach a Dag  in the heap.\\
			$\p_{\subC} == \true \Rightarrow \p$ can reach a Cycle  in the heap. 
	\end{center}
	\pause
	\item The boolean functions consist of the values from:
		\begin{itemize}
			\item $D_F$
			\item $I_F$ 
			\item Boolean variables
		\end{itemize}	
	\pause
	\item An Example:
	$$
			\p_{\subC} = f_{\p\q} \wedge \num{D_F[\q,\p]} \geq 1
	$$		
	\end{itemize}
}

\frame
{
	\frametitle{\subsecname}
	\cmt{\begin{itemize}
	\item The shape of \p, \p.\shape, can be obtained by evaluating the functions for the attributes $\p_{\subC}$ and
          $\p_{\subD}$, and using following Table.}
		  \begin{table}
			\caption{Determining shape from boolean
			  attributes}
			\begin{center}
			%\begin{tabular*}{0.75\textwidth}{@{\extracolsep{\fill}} cccc|c }
			\begin{tabular}{|cc|cc|c| }
			\hline
			$\p_{\subC}$ &$\quad$ & $\p_{\subD}$ &$\quad$& $\p.\shape$ \\ 
			\hline
			\true  && Don't Care  && Cycle        \\ 
			\false  && \true          && DAG    \\ 
			\false  && \false          && Tree   \\ 
			\hline
			\end{tabular}
			\end{center}
		\end{table}

	\cmt{\end{itemize}}
}

\subsection{Analysis Framework}
\frame
{
	\frametitle{\subsecname}
	
	\begin{itemize}
	\item Forward data flow analysis framework. \cmt{ where the data flow values are the $D_F$, $I_F$ and 
	the boolean variables.}
	\pause
	\item On demand evaluation of boolean functions.
	\cmt{We do not evaluate the boolean functions immediately, but associate
the unevaluated functions with the statements. When we want to find out the shape at a given
statement, only then we evaluate the function using the $D_F$ and $I_F$ matrices, and the values of
boolean variables at that statement.}
	\cmt{\pause
	\item Field connectivity information is updated directly by the statement.}
	\end{itemize}
}

\subsection{Motivational Example Revisited}
\frame
{
	\frametitle{\subsecname}
	
\begin{figure}
  \begin{center}
  
    \scalebox{.80}{\begin{tabular}{ |@{}l@{ }|@{}c@{}|@{}c@{}|@{}c@{}|@{}c@{}|  }
   \hline
   \multirow{2}{*}{Statements} & \multirow{2}{*}{Heap Structure} & \multirow{2}{*}{Boolean Functions} &  \multicolumn{2}{@{}c@{}|}{Shape Inference}  \\
   \cline{4-5} 
   &  &  &   p.shape & q.shape  \\
   \hline
	 &&&& \\
    {\tt S1.} \ttf{p = malloc()} & \multirow{2}{*}{\input{motiv_1_fig_S1.tex}} &  
																				$\begin{array}{@{}c@{}}
																					p_{\subC} = \false  \qquad q_{\subC} = \false \\ 
																					p_{\subD} = \false  \qquad q_{\subD} = \false 
																				\end{array}$
																				& \multirow{2}{*}{Tree} & \multirow{2}{*}{Tree} \\
    {\tt S2.} \ttf{q = malloc()} &    &&&\\
	 &&&& \\
    \hline
	\pause
	 &&&& \\
	{\tt S3.} \ttf{p$\rightarrow$f = q} &  \input{motiv_1_fig_S3.tex} & 
																				$\begin{array}{@{}c@{}}
																					p_{\subC} = (f_{\p\q} \wedge \num{D_F[q,p]} \geq 1) \\ \\
																					q_{\subC} = f_{\p\q} \wedge \num{D_F[q,p]} \geq 1 \\ \\
																					p_{\subD} = f_{\p\q} \wedge \num{I_F[\p,\q]} > 1 \\ \\
																					f_{\p\q} = \true
																				\end{array}$
																	  & Tree & Tree\\
	 &&&& \\
     \hline
	\pause
	 &&&& \\
	{\tt S4.} \ttf{p$\rightarrow$h = q} &  \input{motiv_1_fig_S4.tex} & 
																				$\begin{array}{@{}c@{}}
																					\p_{\subD} = (f_{\p\q} \wedge \num{I_F[\p,\q]} > 1) \vee (h_{\p\q} \wedge \num{I_F[\p,\q]} > 1) \\ \\
																					f_{\p\q} = \true \qquad h_{\p\q} = \true
																				\end{array}$
																	  & Dag & Tree\\
	 &&&& \\
     \hline
    \end{tabular}}
  \end{center}
\end{figure}
}
\frame
{
	\frametitle{\subsecname}
	
\begin{figure}
  \begin{center}
  
    \scalebox{.80}{\begin{tabular}{ |@{}l@{ }|@{}c@{}|@{}c@{}|@{}c@{}|@{}c@{}|  }
   \hline
   \multirow{2}{*}{Statements} & \multirow{2}{*}{Heap Structure} & \multirow{2}{*}{Boolean Functions} &  \multicolumn{2}{@{}c@{}|}{Shape Inference}  \\
   \cline{4-5} 
   &  &  &   p.shape & q.shape  \\
     \hline
	 &&&& \\
	{\tt S5.} \ttf{q$\rightarrow$g = p} &  \input{motiv_1_fig_S5.tex} 	& 
																				$\begin{array}{@{}c@{}}
																					\p_{\subC} = (g_{\q\p} \wedge \num{D_F[\p,\q]} \geq 1) \\ \\
																					\p_{\subD} = (f_{\p\q} \wedge \num{I_F[\p,\q]} > 1) \vee (h_{\p\q} \wedge \num{I_F[\p,\q]} > 1) \\ \\
																					f_{\p\q} = \true \qquad h_{\p\q} = \true \qquad g_{\q\p} = \true 
																				\end{array}$
	
																		& Cycle & Cycle\\
	 &&&& \\
     \hline
	 \pause
	 &&&& \\
	{\tt S6.} \ttf{q$\rightarrow$g = null} &  \input{motiv_1_fig_S6.tex} 	& 
																				$\begin{array}{@{}c@{}}
																					\p_{\subC} = (g_{\q\p} \wedge \num{D_F[\p,\q]} \geq 1) \\ \\
																					\p_{\subD} = (f_{\p\q} \wedge \num{I_F[\p,\q]} > 1) \vee (h_{\p\q} \wedge \num{I_F[\p,\q]} > 1) \\ \\
																					f_{\p\q} = \true \qquad h_{\p\q} = \true \qquad g_{\q\p} = \false 
																				\end{array}$
	
																		& Dag & Tree\\
	 &&&& \\
     \hline
	 \pause
	 &&&& \\
	{\tt S7.} \ttf{p$\rightarrow$h = null} &  \input{motiv_1_fig_S7.tex} 	& 
																				$\begin{array}{@{}c@{}}
																					\p_{\subC} = (g_{\q\p} \wedge \num{D_F[\p,\q]} \geq 1) \\ \\
																					\p_{\subD} = (f_{\p\q} \wedge \num{I_F[\p,\q]} > 1) \vee (h_{\p\q} \wedge \num{I_F[\p,\q]} > 1) \\ \\
																					f_{\p\q} = \true \qquad h_{\p\q} = \false \qquad g_{\q\p} = \false 
																				\end{array}$
	
																		& Tree & Tree\\
	 &&&& \\
     \hline
    \end{tabular}}
  \end{center}
\end{figure}
}

\section{Properties}
\subsection{Need for Boolean Variables}
\frame
{
	\frametitle{\subsecname}

	\input{properties_fig_1.tex}
	\[ \upath \quad=\quad \epsilonset \cup \bigcup_{f\in\fields} \{f^{\drct},
f^{\indrct\infty}\} \]
	After {\tt S2},
	\begin{eqnarray*}
   \mb{r}_{\subD} = (f_{pq} \wedge (|\IFM{r}{q}| > 1))
   	&\qquad& f_{pq} = \true
 	\end{eqnarray*}
}

\subsection{Storage Requirement}
\frame
{
	\frametitle{\subsecname}
	\begin{eqnarray*}
\mbox{Space requirement for } D_F &:&  O(n^{2}* m) \\
\mbox{Space requirement for } I_F &:& O(n^{2} * m^2) \\
\mbox{Height and width of the expression tree} &:& \mbox{Polynomial in number} \\
												&& \mbox{of pointer instructions} \\
												&& \mbox{executed}
\end{eqnarray*}
}

\section{Test Results}
\subsection{Inserting an internal node in a singly linked list.}
\frame
{
	\frametitle{\subsecname}

	\newcommand{\on}{p} 
	\newcommand{\nn}{r}

\begin{figure}
\centering
\scalebox{0.80}{
\begin{tabular}{@{}ll@{}}
  {\small\tt
    \begin{tabular}[b]{l}
      S1. \on$\rightarrow$f = q; \\
      S2. \nn = malloc(); \\
      S3. \nn$\rightarrow$f = q;  \\
      S4. \on$\rightarrow$f = \nn;  
    \end{tabular}
  } &
  \scalebox{0.80}{
    \begin{tabular}[b]{|c||c|c|c|}
      \hline
          {\bf After} & Actual Shape & Field Insensitive
          Analysis & Field Sensitive Analysis \\ 
          \hline \hline
	  {\tt S1}       & Tree		  & Tree    & Tree \\ \hline
	  {\tt S2}       & Tree		  & Tree    & Tree \\ \hline
	  {\tt S3}       & Tree		  & Tree    & Tree \\ \hline
	  {\tt S4}       & Tree		  & DAG(at {\tt \on})    & Tree \\ \hline
    \end{tabular}
  } \\
  \multicolumn{2}{c}{\scalebox{1.00}{(a) Insertion of an
      internal node in a singly linked list }}
	  \end{tabular}}
	  \end{figure}
	  Boolean function generated after {\tt S4}:
\begin{eqnarray*}
  \p_{\subD} &=&   (f_{\p\q} \wedge (\num{I_F[\p\q]} > 1)) 
  \vee  (f_{\p\myr} \wedge (\num{I_F[\p\myr]} > 1)), \\
  f_{\p\myr} &=& \true, \qquad\qquad  f_{\p\q} = \false \enspace. 
\end{eqnarray*}
}
\subsection{Recursively Swapping Binary Tree.}
\frame
{
	\frametitle{\subsecname}

	\newcommand{\on}{p} 
	\newcommand{\nn}{r}

\begin{figure}
\centering
\scalebox{0.80}{
\begin{tabular}{@{}ll@{}}
{\small \tt
    \begin{tabular}[b]{l}
      mirror(tree T) \{ \\
      S1.    L  = T->left; \\
      S2.    R = T->right; \\
      S3.    mirror(L); \\
      S4.    mirror(R); \\
      S5.    T->left = R; \\
      S6.    T->right = L; \\
      \}  
    \end{tabular}
  } &
  \scalebox{0.80}{
    \begin{tabular}[b]{|c||c|c|c|}
      \hline
          {\bf After} & Actual Shape & Field Insensitive
          Analysis & Field Sensitive Analysis \\ 
          \hline \hline
	  {\tt S1}       & Tree		  & Tree    & Tree \\ \hline
	  {\tt S2}       & Tree		  & Tree    & Tree \\ \hline
	  {\tt S5}       & Dag (at T)	  & Dag (at T)    & Dag (at T) \\ \hline
	  {\tt S6}       & Tree		  & Dag (at T)    & Tree \\ \hline
    \end{tabular}
  } \\
  \multicolumn{2}{c}{\scalebox{1.00}{(c) Computing mirror
      image of a  binary tree.}}
\end{tabular}}
\end{figure}

Boolean functions generated after {\tt S5}:
\begin{eqnarray*}
  T_{\subD} &=&   (left_{T,R} \wedge \num{I_F[T,R]} > 1) \\ 
  left_{T,R} &=& \true 
\end{eqnarray*}
}

\section{Future Work}
\subsection{Future Work}
\frame
{
	\frametitle{\subsecname}
	\begin{itemize}
	\item We use a very simple inter procedural framework to handle
function calls, that computes safe approximate summaries to
reach fix point .  Our next challenge is to develop a better
inter procedural analysis to handle function calls more
precisely. 
	\item To extend our shape analysis
technique to handle more of frequently occurring programming
patterns.
     \item To develop a prototype model using
GCC framework to show the effectiveness on large
benchmarks. 
	\end{itemize}
}

\begin{frame}[plain]

	\begin{center}
	\Huge{\emph{\blue Thank You }}
	\end{center}
\end{frame}
}
\end{document}

