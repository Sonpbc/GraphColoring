\documentclass[10pt,twoside]{article}
\usepackage{url}

\newcommand{\doctitle}{%
Sudoku Solver}
\newcommand{\cmnt}[1]{}

\pagestyle{myheadings}
\markboth{\hfill\doctitle}{\doctitle\hfill}

\bibliographystyle{siam}

\addtolength{\textwidth}{1.00in}
\addtolength{\textheight}{1.00in}
\addtolength{\evensidemargin}{-1.00in}
\addtolength{\oddsidemargin}{-0.00in}
\addtolength{\topmargin}{-.50in}

\hyphenation{in-de-pen-dent}

%\title{\textbf{\doctitle}\\
\title{\textbf{ CS598 Project Proposal: Parallel Sudoku Solver}}

\author{Sandeep Dasgupta\thanks{Electronic address: \texttt{sdasgup3@illinois.edu}}
\qquad Tanmay Gangwani\thanks{Electronic address: \texttt{gangwan2@illinois.edu}}
\qquad  Mengjia Yan\thanks{Electronic address:
\texttt{myan8@illinois.edu}}} 

\begin{document}

\thispagestyle{empty}

\maketitle

  Sudoku puzzles consists of partially filled matrix N×N. The algorithm needs
  to fill the blank positions with values 1 through N such that no number is
  repeated on each of the N rows, N columns or the squares of $\sqrt{N} \times
  \sqrt{N} $ cells that split the original matrix.  The goal will be to solve
  the grid using parallelization strategies provided by Charm++.

  From a mathematical perspective, it has been proven that the total number of
  valid Sudoku grids, for $N = 9$,  is $6,670,903,752,021,072,936,960$ or
  approximately $6.671 \times 10^{21}$.  Trying to populate all these grids is
  itself a difficult problem because of the huge number.  This large number
  also directly eliminates the possibility of solving the puzzle with brute
  force technique in a reasonable amount of time. Therefore, a method for
  solving the puzzle quickly will be derived that takes advantage of some
  "logical' properties of Sudoku to reduce the search space and optimize the
  running time of the algorithm. Our goal off course, is to allow for that
  algorithm to take advantage of the manycore architecture to further reduce
  its running time.  

\cmnt{
  On many occasions the humanistic algorithm returns a board with unfilled cells
  left, where they apply some brute force methods which involves some  heuristics
  to choose a value for a cell and see if that helps to solve a puzzle and many a times 
  that end up being wrong guess. To work around the problem, “backtracking” 
  ( or simply, undoing the puzzle 
      to a previous state just before the wrong guess is made and to  continue from there 
      by taking a different guess). Checkpoint/Restart
}
  
  



%\nocite{*}
%\bibliography{CS598_project_proposal}

\end{document}
